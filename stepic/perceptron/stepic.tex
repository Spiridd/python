\documentclass[14pt]{extarticle}

\usepackage{geometry}
\usepackage{amsmath,amsthm,amssymb}
\usepackage[utf8]{inputenc}
\usepackage[T1,T2A]{fontenc}
\usepackage{bold-extra}
\usepackage[english,russian]{babel}
\usepackage{indentfirst}
\usepackage{graphicx}
\graphicspath{ {images/} }
\usepackage{float}
\usepackage{listings}
\usepackage{lmodern}
\usepackage{appendix}
\usepackage{braket}
\usepackage{cite}
\usepackage[nottoc,numbib]{tocbibind}

\geometry{
a4paper,
left = 30mm,
right = 10mm,
bottom = 20mm,
top = 20mm,
}
\renewcommand{\rmdefault}{ftm} % TimesNewRoman
\renewcommand{\baselinestretch}{1.5} 
\begin{document}

Рассмотрим произвольную бесконечно дифференцируемую функцию $ J(x) $, $ x = (x_1, ..., x_n) $ --- вектор параметров, и произвольное приращение аргумента $ \Delta x $. Пользуясь формулой Тейлора для функции многих переменных, запишем:
\begin{align*}
J \left( x + \Delta x \right) - J \left( x \right)
= J \left( x \right) + \left( \Delta x, \nabla \right) J \left( x \right) + o \left( \Vert \Delta x \Vert \right) - J \left( x \right) = \\
= \left( \Delta x, \nabla \right) J \left( x \right) + o \left( \Vert \Delta x \Vert \right)
= \left( \Delta x, \nabla J \right) + o \left( \Vert \Delta x \Vert \right),
\end{align*}
где $ \Vert \Delta x \Vert = \sqrt{\sum \limits_{i=1}^n \left( \Delta x_i \right)^2} $.
Мы получили, что при достаточно малом приращении $ \Delta x $ изменение функции приближённо равно скалярному произведению приращения аргумента и градиента функции: $ \left( \Delta x, \nabla J \right) $. Далее будем считать, что приращение аргумента имеет фиксированную достаточно малую длину $ \epsilon $: $\Vert \Delta x \Vert = \epsilon $.

Для произвольного приращения аргумента верно следующее (использовано неравенство Коши-Буняковского):
\begin{equation*}
J(x + \Delta x) - J(x) 
= (\nabla J, \Delta x) 
\le \Vert \nabla J \Vert \cdot \Vert \Delta x \Vert
= \Vert \nabla J \Vert \cdot \epsilon. 
\end{equation*}
Рассмотрим приращение вдоль градиента: 
$$ \Delta x' = \frac{\nabla J}{\Vert \nabla J \Vert} \epsilon. $$
Тогда в силу определения скалярного произведения получим:
\begin{equation*}
J(x + \Delta x') - J(x) 
= \left( \nabla J, \Delta x' \right)
= \left( \nabla J, \frac{\nabla J}{\Vert \nabla J \Vert} \epsilon \right) 
= \Vert \nabla J \Vert \cdot \epsilon.
\end{equation*}
Очевидно, что $ \Delta x' $ доставляет максимум приращению функции, т.е. градиент задаёт направление наибольшего роста функции:
\begin{equation*}
J(x + \Delta x) - J(x) \le \Vert \nabla J \Vert \cdot \epsilon 
= J(x + \Delta x') - J(x).
\end{equation*}


\end{document}
